\documentclass[a4paper, 11pt]{article}

%% packages
\usepackage{fullpage} % changes the margin
\usepackage{hyperref} % Links
\usepackage[utf8]{inputenc}
\usepackage{lmodern}
\usepackage{amsfonts}
\usepackage{amsthm}
\usepackage{amsmath}
\usepackage{braket}
\usepackage{graphicx}
%%

%% environments
\theoremstyle{definition}
\newtheorem{definition}{Definition}
\newtheorem{examples}{Example}
\newtheorem{exercises}{Exercises}
\newtheorem{exercise}{Exercise}
\newtheorem{postulate}{Postulate}
\usepackage[linewidth=1pt]{mdframed}
%%

%% config
\date{}
\linespread{1.15}
%%

\begin{document}

\allowdisplaybreaks[2]
\title{Quantum Computing @ MEF \\ \small{TPC-2}}
\author{Renato Neves \\ \scriptsize
  \href{mailto:nevrenato@di.uminho.pt}{nevrenato@di.uminho.pt}}
\maketitle

\noindent
Consider the Boolean formula $\varphi = A \wedge (\neg B \vee C)$. Can
we assign values to variables $A$, $B$ and $C$ such that $\varphi$ is
true? Such a question is an instance of the \emph{Satisfiability
  Problem}~\cite{schoning13}. The latter appears frequently across
different domains, from theorem proving to software verification,
cryptography, and artificial intelligence -- it is, in fact, one of
the most discussed problems in Computer Science.

The goal of this assignment is to write an essay (around 5 pages) that
discusses how Grover's algorithm~\cite{nielsen16} can be used to
tackle the Satisfiability problem. Among other things, we will value
essays whose claims and ideas are tested and illustrated via
implementations in \texttt{Qiskit}.

\begin{mdframed}
  What to submit: The report in \texttt{PDF} and if applicable your
  \texttt{Qiskit} implementations. Please send by email
  (\texttt{nevrenato@di.uminho.pt}) a unique zip file with the name
  ``\texttt{qc2122-N.zip}'', where ``\texttt{N}'' is your student
  number.  The subject of the email should be ``\texttt{qc2122 N
    TPC-2}''.
\end{mdframed}



%% Bibliography
\bibliographystyle{alpha}
\bibliography{biblio}

\end{document}