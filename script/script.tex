\documentclass[a4paper, 11pt]{article}

%% packages
\usepackage{fullpage} % changes the margin
\usepackage{hyperref} % Links
\usepackage[utf8]{inputenc}
\usepackage{lmodern}
\usepackage{amsfonts}
\usepackage{amsthm}
\usepackage{amsmath}
\usepackage{braket}
%%

%% macros
\newcommand{\complex}{\mathbb{C}}
\newcommand{\vecs}{\mathcal{V}}
\newcommand{\id}{\mathrm{id}}
%% environments
\theoremstyle{definition}
\newtheorem{definition}{Definition}
\newtheorem{examples}{Example}
\newtheorem{exercises}{Exercises}
\newtheorem{exercise}{Exercise}
\newtheorem{postulate}{Postulate}
%% config
\date{}
\linespread{1.15}
%%

\begin{document}

\title{Quantum Computing @ MEF \\ \large Background}
\author{Renato Neves \\ \scriptsize
  \href{mailto:nevrenato@di.uminho.pt}{nevrenato@di.uminho.pt}}
\maketitle

\section{Quantum States}

Models of computation traditionally put at center stage a notion of
state and a corresponding notion of state
transition~\cite{bruni17}. In the quantum world, states usually
involve superpositions, angles, and lengths; or in other words, they
involve aspects related to geometry. This suggests us to get familiar
with both vector spaces and the more refined notion of an inner
product space. We will also need to delve deep into the inner workings
of maps between vector spaces and maps between inner product spaces,
both intuitively giving rise to the notion of a quantum state
transition.

\subsection{Vector spaces}

Let $\complex$ denote the set of complex numbers.

\begin{definition}[Vector Space]
  A vector space (over the complex numbers)~\footnote{In this course we will
    only consider vector spaces over the complex numbers.} is a set
  $V$ together with an `addition' operation $+ : V \times V \to V$, a
  `multiplication' operation $\cdot : \complex \times V \to V$, a
  `zero' element $0 \in V$, and an `inverse' operation $- : V \to V$
  such that the following equations hold:
  \begin{align*}
    v + (u + w) & = (v + u) + w & v + u & = u + v  \\
    v + 0 & = v & v + (-v) & = 0 \\
    (s r) \cdot v & = s \cdot (r \cdot v) & 1 \cdot v & = v  \\
    s \cdot (v + u) & = s \cdot v + s \cdot u & (s + r) \cdot v & = s \cdot v + r \cdot u
  \end{align*}
\end{definition}

To keep notation simple we will often omit the dot of the scalar
multiplication, i.e. we will write expressions $s \cdot v$ simply
as $s v$.

\begin{examples}
  Note that the complex numbers themselves form a vector space, and
  that the set $\complex^2$ of pairs of complex numbers also forms a
  vector space. Recall that this last set underlies the mathematical
  representation of a qubit -- i.e. the unit in quantum information
  (later on we will see that our notion of quantum state is based on
  sequences of qubits).
\end{examples}


\begin{exercise}
  Show that for any finite set $n$ we can build a vector space
  $[n,\complex]$ over the complex numbers. Show also that the set
  $\mathsf{Mat}_\complex(n,m)$ of matrices with $n$ lines and $m$
  columns and whose values are complex numbers also forms a vector
  space (hint: observe that matrices can be given a functional
  representation).
\end{exercise}

\begin{definition}[Linear maps a.k.a. linear operators or simply operators]
  Consider two vector spaces $V$ and $W$. A linear map $f : V \to W$
  is a function that satisfies the equations,
  \begin{align*}
    f(v_1 + v_2) = f (v_1) + f(v_2) \hspace{2cm}
    f (s v) = s  f(v)
  \end{align*}
  We call $f$ a \emph{linear isomorphism} or simply isomorphism if it
  is bijective.
  %% check the correspondence in cats
  When such is the case, we say that $V$ and $W$ are isomorphic to each
  other (i.e. essentially the same), in symbols $V \simeq W$.
\end{definition}

\begin{exercise}
  Show that the identity map $\id : V \to V$ is linear. Additionally show
  that if $f : V \to W$ and $g : W \to U$ are linear maps then their
  composition $g \cdot f : V \to U$ is also a linear map.
\end{exercise}

\begin{exercise}
  Consider a vector space $V$. Show that linear maps
  $f : \complex \to V$ are in one-to-one correspondence with the
  elements of $V$.
\end{exercise}


Another important concept for the notion of quantum state and quantum
state transition is that of tensoring. In essence, it allow us to
mathematically represent multiple qubits (instead of working with just
one) and thus to increase the computational power at
hand\footnote{This is actually critical for taking full advantage of
  quantum computing.}.

\begin{definition}[Tensor]
  Let $V$ and $W$ be two vector spaces. Their tensor, denoted by
  $V \otimes W$, is the vector space consisting of all linear
  combinations $\sum_{i \leq n} s_i  (v_i \otimes w_i)$ with
  $v_i \in V$, $w_i \in W$, that satisfies the equations,
  \begin{align*}
   (v \otimes w) + (u \otimes w) & = (v + u) \otimes w &
   (w \otimes v) + (w \otimes u) & = w \otimes (v + u) \\
   s  (v \otimes w) & = (s v) \otimes w &
   s  (v \otimes w) & = v \otimes (s w)
  \end{align*} 
\end{definition}

\begin{exercise}
  Show that from linear maps $f : V \to V'$ and $g : W \to W'$ we can
  define a new linear map
  $f \otimes g : V \otimes W \to V' \otimes W'$. Show that
  $(f \otimes g) \cdot (f' \otimes g') = (f \cdot f' ) \otimes (g
  \cdot g')$. Prove that there exist linear isomorphisms
  $V \otimes W \simeq W \otimes V$ and $V \otimes \complex \simeq V$.
\end{exercise}

\begin{exercise}
  Show that the map $\Delta : V \to V \otimes V$ defined by
  $\Delta(v) = v \otimes v$ is \emph{non-linear}. How is this related
  to the no-cloning theorem?
\end{exercise}

Another concept that we will use extensively is that of a basis.

\begin{definition}[Basis]
  A basis for a vector space $V$ is a set $B \subseteq V$ of vectors that
  respects the following conditions:
  \begin{itemize}
  \item for every $v \in V$, we can find $v_1,\dots,v_n \in B$ and
    $s_1,\dots,s_n \in \complex$ such that
    $\sum_{i \leq n} s_i v_i = v$
  \item for every sequence of vectors $v_1,\dots,v_n \in B$ and
    sequence of complex numbers $s_1,\dots,s_n \in \complex$ if
    $\sum_{i \leq n} s_i  v_i = 0$ then $s_i = 0$ for all
    $i \leq n$.
  \end{itemize}
\end{definition}

\begin{examples}
  The set $\{ 1 \}$ is a basis for $\complex$  and the set
  $\{(1,0),(0,1)\}$ is a basis for $\complex^2$.
\end{examples}

Let $B$ be a basis for a vector space $V$. If $B$ has $n$ elements we
say that $V$ is $n$-dimensional. If $B$ is finite we say that $V$ is
\emph{finite-dimensional}.

In this course we are primarily interested in finite-dimensional
vector spaces. Intuitively, this is justified by the fact we will only
need to work with finite numbers of qubits at a time. From now on all
vector spaces that we consider are finite-dimensional.


\begin{exercise}
  Let $n$ be a natural number and $\complex^n$ be the vector space of
  $n$-tuples of complex numbers. Present a basis for $\complex^n$ and
  subsequently indicate its dimension. Next let
  $\mathsf{Mat}_\complex(n,m)$ be the vector space of matrices with
  $n$ lines and $m$ columns and whose values are complex
  numbers. Present a basis for this space and subsequently indicate
  its dimension.
\end{exercise}


\begin{exercise}
  Consider a linear map $f : V \to W$ and let $B$ be a basis for
  $V$. Show that this map is \emph{uniquely determined} by the way it
  maps the elements in the basis of $V$. Moreover, show that a
  function $B \to W$ mapping elements in the basis of $V$ to $W$
  induces a linear map of type $V \to W$.
\end{exercise}


\begin{exercise}
  Show that any vector space $V$ with dimension $n$ is isomorphic to
  the vector space $\complex^n$.
\end{exercise}

Matrices are often a convenient way of expressing states and computing
state transitions. In our case we are fortunate enough that states
$\complex \to V$ and linear maps $V \to W$ can be equivalently
represented as matrices (whose dimensions depend on those of $V$ and
$W$). We briefly describe how this works next.  Let $V$ and $W$ be
vector spaces, $\{b_1,\dots,b_n\}$ a basis for $V$ and
$\{c_1,\dots,c_m\}$ a basis for $W$. Consider then a linear map
$f : V \to W$ and observe that for every $i \leq n$ we have
$f(b_i) = \sum_{j \leq m} s_{ij} c_j$ for some
$s_{i1}, \dots, s_{im} \in \complex$. We obtain a matrix
$M \in \mathsf{Mat}_\complex(m,n)$ by setting $M_{ji} = s_{ij}$.
Conversely, consider a matrix $M \in \mathsf{Mat}_\complex(m,n)$. We
obtain a linear map $f : V \to W$ by setting
$f(b_i) = \sum_{j \leq m} M_{ji} c_j$.

\begin{exercise}
  Show that the two operations described above (for switching between
  linear maps and their matrix representation) are inverse of each
  other.
\end{exercise}

\begin{exercise}
  What is the matrix corresponding to the linear map
  $f : \complex^2 \to \complex^2$ defined by $f(1,0) = (0,1)$ and
  $f(0,1) = (1,0)$? What is the matrix corresponding to the linear map
  $f : \complex^2 \to \complex^2$ defined by
  $f(1,0) = \frac{1}{\sqrt{2}}(1,0) + \frac{1}{\sqrt{2}}(0,1)$ and
  $f(0,1) = \frac{1}{\sqrt{2}}(1,0) - \frac{1}{\sqrt{2}}(1,0)$?
\end{exercise}

In the sequel, let $M : n \to m$ denote a matrix with $n$ lines, $m$
columns, and whose values are complex numbers. Also for two matrices
$M : n \to m$ and $N : m \to o$, let $M N : n  \to  o$
denote the matrix multiplication of $M$ with $N$. Finally, given a
linear map $f : V \to W$ such that $V$ and $W$ have dimension $n$ and
$m$, respectively, let $M_f : m \to n$ denote the corresponding matrix.
  
\begin{exercise}
  Show that elements of $v \in V$ are in one-to-one correspondence
  with elements $M_v$ of $\mathsf{Mat}_\complex(n,1)$. Then show that
  $M_f M_g = M_{g \cdot f}$.
\end{exercise}


\begin{exercise}
  Let $B \subseteq V$, $C \subseteq W$ be bases for vector spaces $V$
  and $W$, respectively.  Show that the set
  $\{ b \otimes c \mid b \in B, c \in C \}$ is a basis for
  $V \otimes W$. Then show that
  $\complex^n \otimes \complex^m \simeq \complex^{n m}$.
\end{exercise}

Consider matrices $M : n \to m$ and $N : o \to p$. Their tensor
$M \otimes N : n \cdot o \to m \cdot p$ (also called Kronecker
product) is defined by,
\begin{align*}
  M \otimes N =
  \begin{bmatrix}
    M_{1,1} \cdot N, & \dots, &  M_{1,m} \cdot N \\
    \vdots & \vdots & \vdots \\
    M_{n,1} \cdot N, & \dots, &  M_{n,m} \cdot N
  \end{bmatrix}
\end{align*}
Note that $M_{f \otimes g} = M_f \otimes M_g$.

\begin{exercise}
  For a given matrix $M : n \to m$, let us use $M^\ast : n \to m$ to
  denote the matrix such that $M_{ij}^\ast = M_{ij}$, $M^T : m \to n$
  to denote the transpose of $M$, and $M^\dagger : m \to n$ to denote
  $(M^T)^\ast$, i.e. the conjugate transpose of $M$. Show that the
  following equations hold.
  \begin{align*}
    (M \otimes N)^\ast = M^\ast \otimes N^\ast \hspace{1cm}
    (M \otimes N)^T = M^T \otimes N^T \hspace{1cm}
    (M \otimes N)^\dagger = M^\dagger \otimes N^\dagger
  \end{align*}
\end{exercise}


\subsection{Inner product spaces}

Recall that for some complex number $c$, the expression $c^\ast$
denotes the complex conjugate of $c$.

\begin{definition}[Inner product space] An inner product space is a
  vector space $V$ equipped with a function
  $\langle \cdot,\cdot \rangle : V \times V \to \complex$ (the inner
  product) that satisfies the conditions,
  \begin{align*}
    \left \langle v, \sum_{i \leq n} s_i v_i \right \rangle
    & = \sum_{i \leq n} s_i \cdot \langle v,v_i \rangle
    & \langle v,w \rangle  & = \langle w,v \rangle^\ast \\
    \langle v,v \rangle & \geq 0 & 
    \langle v,v \rangle & = 0 \text{ entails } v = 0
  \end{align*}
\end{definition}

\begin{exercise}
  Let $n$ be a natural number.  Show that the vector space
  $\complex^n$ becomes an inner product space when equipped with the
  function $\langle \cdot, \cdot \rangle
  : \complex^n \times \complex^n \to \complex$
  defined by,
  \begin{align*}
   \langle (a_1,\dots,a_n),(b_1,\dots,b_n) \rangle = \sum_{i \leq n} a_i^\ast b_i 
  \end{align*}
\end{exercise}

Every inner product space $V$ induces a norm
$\| \cdot \| : V \to [0,\infty)$ defined by
$\| v \| = \sqrt{\langle v,v \rangle}$. The mathematical
representation of the state of $n$-qubits is a vector $v \in \complex^{2^n}$
with norm $\| v \| = 1$.

\begin{exercise}[Vector normalisation]
  Let $v \in V$ be a vector. Show that,
  \begin{align*}
   \left \| \frac{v}{\| v \|} \right \| = 1 
  \end{align*}

\end{exercise}

\begin{definition}[Orthonormal basis]
  Two vectors $v,w \in V$ are said to be orthogonal to each other if
  $ \langle v,w \rangle = 0$.  A basis $B$ for an inner product space
  $V$ is called orthonormal if all elements of $B$ have norm $1$ and
  are orthogonal to each other.
\end{definition}

\begin{exercise}
  Show that the basis $\{ (1,0), (0,1) \}$ for $\complex^2$ is
  orthonormal.
\end{exercise}

\begin{definition}[Tensor]
  Let $V$ and $W$ be two inner spaces. Their tensor, denoted by
  $V \otimes W$, is the tensor of $V$ and $W$ as vector spaces
  equipped with the function,
  \begin{align*}
    \left \langle  \sum_{i \leq n} s_i (v_i \otimes w_i), \sum_{j \leq m} s_j (v_j \otimes w_j)
    \right \rangle = \sum_{i \leq n, j \leq m} s_i^\ast s_j \cdot \langle v_i,v_j \rangle
    \cdot \langle w_i,w_j \rangle
  \end{align*}
\end{definition}

\begin{exercise}
  Let $B \subseteq V$, $C \subseteq W$ be orthonormal bases for inner
  product spaces $V$ and $W$, respectively.  Show that the set
  $\{ b \otimes c \mid b \in B, c \in C \}$ is an orthonormal basis
  for $V \otimes W$.
\end{exercise}

The notion of reversible computation in quantum computing is related
to the following fact.  Consider a linear map $f : V \to W$ between
inner product spaces $V$ and $W$. There exists a unique linear map
$f^\dagger : W \to V$ such that for all $v \in V$ and $w \in W$ the
equation,
\begin{align*}
  \langle f (v), w \rangle = \langle v, f^\dagger (w) \rangle
\end{align*}
holds. This map is often called the Hermitian conjugate (or adjoint)
of $f$ -- its matrix representation is precisely the conjugate
transpose of the matrix $M_f$. A particularly important family of
operations that builds on this notion is that of unitary maps.

\begin{definition}[Unitary maps]
  Let $V$ be an inner product space. A linear map $f : V \to V$ is
  called unitary if $f^{-1}$ exists and $f^{-1} = f^\dagger$.  An
  equivalent (and insightful) characterisation of unitary maps tells
  that they are precisely those that satisfy the equation,
  \begin{align*}
    \| v \| = \| f(v) \|
  \end{align*}
  which in the particular case of $V = \complex^{2^n}$ means that
  quantum states are always mapped to quantum states (and not
  something else).
\end{definition}

\begin{exercise}
  Show that the following two maps are unitary:
  \begin{itemize}
  \item $f : \complex^2 \to \complex^2$ defined by $f(1,0) = (0,1)$ and
    $f(0,1) = (1,0)$.
  \item $g : \complex^2 \to \complex^2$ defined by
  $g(1,0) = \frac{1}{\sqrt{2}}(1,0) + \frac{1}{\sqrt{2}}(0,1)$ and
  $g(0,1) = \frac{1}{\sqrt{2}}(1,0) - \frac{1}{\sqrt{2}}(1,0)$.
  \end{itemize}
\end{exercise}

\begin{exercise}
  Prove that if two linear maps are unitary then their tensor is also
  unitary.
\end{exercise}

\begin{postulate}[Quantum state and state transition]
  The state of an \emph{isolated} quantum computer is given by a unit
  vector in the space $\complex^{2^n}$ for some finite number $n$ --
  the number $n$ corresponds to the number of available qubits. State
  transitions arise via unitary maps, more concretely the state of an
  isolated quantum computer changes by an application of a unitary
  map.~\footnote{See a more general version of this postulate in
    Section 2.2 of~\cite{nielsen02}}
\end{postulate}

\section{Quantum Measurement}

In order to render notation more convenient, we will now use $\ket{0}$
and $\ket{1}$ to denote the elements $(1,0)$ and $(0,1)$ in
$\complex^2$, respectively. We extend this notation to any space
$\complex^{2^n}$ by observing that,
\begin{align*}
  \complex^{2^n} \simeq\ \underbrace{\complex^2 \otimes \dots \otimes \complex^2}_
  {n \text{ times} }
\end{align*}
and representing
$\ket{b_1} \otimes \dots \otimes \ket{b_n} \in \complex^{2^n}$ simply
as $\ket{b_1, \dots, b_n}$.

In this course, we will heavily use two maps $m_0$ and $m_1$ of type
$\complex^2 \to \complex^2$ for measuring qubits. The map $m_0$ is
defined by the equations,
\begin{align*}
  m_0 (\ket{0}) = \ket{0} \hspace{2cm} m_0 (\ket{1}) = 0
\end{align*}
and represents the outcome of the qubit measured being at state
$\ket{0}$; the map $m_1$ arises from an analogous reasoning. For the
space $\complex^{2^n}$ we represent the outcome of the $i$-th
qubit being at state $\ket{k}$ by the map,
\begin{align*}
  \underbrace{\id \otimes \dots \otimes \id}_{i-1 \text{ times}} \otimes\ m_k \otimes
  \underbrace{\id \otimes \dots \otimes \id}_{m - i  \text{ times}}
  : \complex^{2^n} \to \complex^{2^n}
\end{align*}
We call maps built in this way and by composing them with one another
`measurement maps'.

\begin{postulate}[Quantum measurement]
  Let $v \in \complex^{2^n}$ be a quantum state and let us consider a
  measurement map $m : \complex^{2^n} \to \complex^{2^n}$. Then the
  probability of  the outcome represented by $m$ is
  $\langle m(v), m(v) \rangle$ and the quantum state after the
  observed outcome is defined by,
  \begin{align*}
    \frac{m(v)}{\| m(v) \|}
  \end{align*}
\end{postulate}

\begin{exercise}
  Let $h : \complex^2 \to \complex^2$ be the unitary map defined by
  the matrix,
  \begin{align*} \frac{1}{\sqrt{2}} \cdot
     \begin{bmatrix}
       1 & 1 \\
       1 & -1
  \end{bmatrix}
  \end{align*}
  What is the probability of the outcome $\ket{0}$ when measuring
  $h(\ket{0})$?
\end{exercise}

\begin{exercise}
  Consider the quantum state,
  \begin{align*}
    \frac{1}{2} \ket{00} + \frac{1}{2} \ket{01} + \frac{1}{2} \ket{10} +
    \frac{1}{2} \ket{11}
  \end{align*}
  What is the probability of the outcome $\ket{0}$ when measuring the
  leftmost qubit? Let us assume that we indeed observed that the
  leftmost qubit is at state $\ket{0}$. What is the probability of
  the outcome $\ket{1}$ when measuring the rightmost qubit?
\end{exercise}

\begin{exercise}
  Consider the quantum state,
  \begin{align*}
    \frac{1}{\sqrt{2}} \ket{00} + \frac{1}{\sqrt{2}} \ket{11}
  \end{align*}
  What is the probability of the outcome $\ket{0}$ when measuring the
  leftmost qubit? What is the probability of the outcome $\ket{1}$
  when measuring the rightmost qubit? Assume that we indeed observed
  that the leftmost qubit is at state $\ket{0}$. Then what is the
  probability of the outcome $\ket{1}$ when measuring the rightmost
  qubit?
\end{exercise}

%% Bibliography
\bibliographystyle{alpha}
\bibliography{biblioTeaching}

\end{document}