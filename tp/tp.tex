\documentclass[a4paper, 11pt]{article}

%% packages
\usepackage{fullpage} % changes the margin
\usepackage{hyperref} % Links
\usepackage[utf8]{inputenc}
\usepackage{lmodern}
\usepackage{amsfonts}
\usepackage{amsthm}
\usepackage{amsmath}
\usepackage{braket}
\usepackage{graphicx}
%%

%% environments
\theoremstyle{definition}
\newtheorem{definition}{Definition}
\newtheorem{examples}{Example}
\newtheorem{exercises}{Exercises}
\newtheorem{exercise}{Exercise}
\newtheorem{postulate}{Postulate}
\usepackage[linewidth=1pt]{mdframed}
%%

%% config
\date{}
\linespread{1.15}
%%

\begin{document}

\allowdisplaybreaks[2]
\title{Quantum Computing @ MEF \\ \small{Assignment}}
\author{Luís Barbosa and Renato Neves \\ \scriptsize
  \href{mailto:nevrenato@di.uminho.pt}{nevrenato@di.uminho.pt}}
\maketitle

\noindent
This assignment is divided in two parts: one is a concrete exercise on
quantum algorithmics; the other, more exploratory, consists of writing
an essay on a topic of Quantum Computing. We detail each of the two
parts next.

\section{Exercise on Quantum Counting}

\noindent
In a previous homework we studied Grover's algorithm as a way of
tackling the \emph{Satisfiability problem}~\cite{schoning13}. In
particular, we saw that the algorithm allows to \emph{compute the
  solutions} of a Boolean formula, such as
$\varphi = A \wedge (\neg B \vee C)$. Note, however, that if one is
only concerned with satisfiability it is unnecessary to compute
solutions and it may be better instead to \emph{only} determine
whether there exists a solution.

Therefore your next task is to use the \emph{Quantum Counting
  algorithm}~\cite{nielsen16} to compute the number of solutions of
$\varphi$ (rather than the solutions themselves) and present the
corresponding circuit in~\texttt{Qiskit}.


\section{Essay on a topic of Quantum Computing}

Write an essay (around 8 pages) on one of the following topics:
\begin{itemize}
\item Error correction;
\item Circuit optimization;
\item Variational methods;
\item Existing quantum programming languages and respective tools;
\item Quantum chemistry;
\item Quantum machine learning;
\item Quantum computing for finance;
\item Quantum Turing machines and other quantum automata;
\item Adiabatic quantum computing;
\item Measurement-based quantum computing;
\item Quantum tomography.  
\end{itemize}
\emph{N.B.} This essay will be presented to the class later on.

\label{sec:ess}
\begin{mdframed}
  What to submit: A report in \texttt{PDF} about Section 1 and 
  respective implementation(s) in \texttt{Qiskit}. A report in
  \texttt{PDF} about Section 2.

  Please send by email (\texttt{nevrenato@di.uminho.pt}) a unique zip
  file with the name ``\texttt{qc2122-N.zip}'', where ``\texttt{N}''
  is your student number.  The subject of the email should be
  ``\texttt{qc2122 N TPC-2}''.
\end{mdframed}




%% Bibliography
\bibliographystyle{alpha}
\bibliography{biblio}

\end{document}